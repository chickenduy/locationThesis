% !TeX root = ../main.tex
% Add the above to each chapter to make compiling the PDF easier in some editors.

\chapter{Introduction}\label{chapter:introduction}
\section{Motivation}

There are currently around 7.7 billion people living on earth with a declining growth rate at now 1.1 percent per year. The United Nations \cite{populationfacts} are predicting that the population growth is going to be falling steadily in the next decades, but despite that, they are projecting the population to increase until 2100. They estimate that the earth will hit approximately 10.9 billion people at the end of the century, illustrated in their diagram in Figure \ref{fig:population}.

\begin{figure}[htpb]
  \centering
  \includegraphics[width=0.8\textwidth]{figures/population.png}
  \caption{Projections for population growth this century \cite{populationdata}}
  \label{fig:population}
\end{figure}

Big cities and metropoles around the globe are inevitably growing denser year by year. According to the UN \cite{populationdata}\cite{growth}, urban areas around the world have been experiencing an influx of people since the start of industrialization. While the same applies to rural areas because of the general growth rate of the population, the increase in people moving into big cities is much higher compared to the countryside.
Depending on how certain factors, such as climate change, autonomous driving, and other technology and politics, evolve in the next decades, it could decelerate or even accelerate growth.

To solve issues stemming from the increased demand in housing, traffic, and infrastructure, urban planning is one of the essential tools. Improving traffic control requires knowledge of the traffic flow and inefficiencies that are causing traffic jams. Knowing local population trends and movement pattern could benefit the design of housing and energy infrastructure. There are only a handful of companies that hold the monopoly on the much-required data, but acquiring them is expensive and may cause problems with privacy protection laws. Publicizing the information without anonymizing it, severely infringes on the privacy of the originator of the data. But the sole act of data anonymization is usually not enough. The use of inference attacks may make it possible to associate data and re-identify the individual, as Sweeney \cite{DBLP:journals/ijufks/Sweene02} has proven.

To solve some of these issues, we can leverage the widespread availability of portable computing power, and create a platform that provides the data on a need-to-know basis. Today, most smartphones are equipped with several sensors, including a GPS sensor, pedometer, and accelerometer, which can be utilized to collect a wide variety of mobility data. Many applications are already taking advantage of those sensors to provide location-based services and games; for instance, Google Maps \cite{maps}, Uber \cite{uber}, and Pokemon GO \cite{pokemon}. "If you're not paying, you're the product" is a quote that is often cited around data-driven applications \cite{newyorktimes}\cite{tedtalk}\cite{arstechnica} and it creates a lot of controversy regarding their kind of applications.

It applies to a lot of free services offered in exchange for your data. Google Maps, for example, provides a free navigation service, while collecting your sensory data, such as GPS and accelerometer, to stay informed about traffic information. But the data is not always used for the improvement of the service, Facebook, for instance, provides a free social media platform for people to connect, in turn, using collected data to sell targeted advertising.

How they use your data is not in your control. So publicizing big anonymous data sets has its limits, and data collection itself is a hot topic in today's media. 

Van Endern \cite{simon} proposed a solution to return control over the data back to the users. He implemented a platform that used the power of crowdsourcing to collect the data and aggregate it on a server on request.

\section{Research Questions}
When collecting data, it is crucial to find a balance between its usability and the privacy of the collecting party. It is also important to keep in mind that there has to be a level of trust in both the data collector and the data analyst to maintain data quality.
\subsection*{RQ1: What are the benefits and drawbacks in van Endern's architecture and what improvements are possible?}
We take a look at van Endern's original idea and implemented architecture. His work lays a great foundation to create a crowdsourcing platform to collect data while protecting the privacy of its users. We will investigate both the advantages and disadvantages of his idea in regards to scalability, security, and privacy. His work shows a minimal viable product on which we can expand further. We will inspect the choices for the design of the Android app and its data collection and take a look at the data flow of the data aggregation and try to locate inefficiencies and deficits and find solutions to the problems.
\subsection*{RQ2: What information do the raw values for steps and activities reveal?}
As van Endern has shown, mean values expose less private information. We will look into the raw data that make up the mean values and analyze the privacy concerns regarding the distribution of the step values and activities values for the participants and if we can use them to link it to other data.
\subsection*{RQ3: How much information does the aggregation of real location data expose?}
Related work 
% Todo: cite all the sources for spatial leakage
has shown that location data are very susceptible to leaking private information. We will try to find ways to collect GPS data and aggregate them without leaking sensitive data. We will research methods to disassociate the mobility data with the individual to provide privacy or anonymity while maintaining statistical relevance.
\subsection*{RQ4: How can peer-to-peer technology be used to improve the platform?}
In van Endern's implemented platform, he uses a central server to communicate with all devices. There have been a lot of advancements in direct communication frameworks, and we will evaluate their feasibility and look into which aspects of the architecture we can take advantage of them to enhance our platform in regards to scalability, security, and privacy-preservation.

\section{Contributions}
Our research has the same structure as van Endern's. He has already laid the groundwork with his bottom-up approach of collecting mobility data using crowdsourcing and the aggregation of some basic data. Similarly,  we only store raw data on the end-devices on which they are generated and only use them to aggregate into a larger data set without identifiers. If necessary, the aggregated data set will be processed to anonymize it before publishing it. Using his idea, we will expand into the aggregation of real location data and analyze them for privacy risks with a field test and a simulated one.

First, we review related work concerning risks and solutions to sensitive data and pertinent anonymization techniques. After that, we examine van Endern's original idea and his final implementation and expand on his architecture. We propose our approach to solve the problems mentioned and increase the types of aggregation. Then, we explain the details of our implementation and the design decisions we made. In the next chapter, we document our test setups and the deployment and evaluate our collected results for information leaks. Finally, we draw a conclusion on van Endern's and our work and discuss further possible improvements and give instructions on how the project can be reproduced.
