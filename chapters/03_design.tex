% !TeX root = ../main.tex
% Add the above to each chapter to make compiling the PDF easier in some editors.

\chapter{Design}\label{chapter:design}
\section{Trust, Usability and Privacy}
In any platform, we expect a certain level trust from each participant, may it be from the service provider or the service consumer. How this trust is created can have many different sources. But the main reason a person or organization can be trusted is because of accountability. Companies with a strong market presence have the public trust since they have been present for a longer period of time and can't just disappear over night, consequently they can be held accountable for their actions. Today, Microsoft, Google, Amazon and Facebook for instance, are trusted to a certain degree. But this trust is not invulnerable. Trust can be damaged by lies and secrecy. Keeping the public uninformed about severe data breaches or selling sensitive data to third parties without their knowledge and permission have strained the public trust in recent years.
% Todo: \cite[Facebook Cambridge Analytica]

Even if trust is in place, we as consumers require privacy. If we can guarantee anonymity however, we can also guarantee privacy, as re-identification should be impossible at that point.
% Todo: \cite[Dictionary]

As chapter 2 has shown, it is a hard task to anonymize big data sets, as quasi-identifiers can be used to infer new information using linking attacks. So one way to achieve anonymity, would be to strip away all attributes that could be used in another data set.

Here we hit a wall with the usability of the data itself. Cynthia Dwork says "de-identified data isn't",
% Todo: \cite[Cynthia Dwork]
meaning that either that the data itself is not de-indentified because of possible reconstruction or the data is not data anymore because it is not useful for analysis. Data itself is only as useful as its relationships.

So there is a need to find a way to find a balance between trust, usability and privacy. Due to the limited scope of the thesis and trust being a very broad research subject, we focus on the privacy of the crowd and usability of their collected data. We assume that all participating devices and servers can be trusted to a certain degree and possible adversaries only have access to the published end result.

\section{Approaches}
As discussed before, there are advantages and disadvantages Simon van Endern's system and explore how we can implement improvements and if they would be feasible with the current state of the art or if another alternative has to be chosen.

\subsection{Simon van Endern}
The proposed architecture implemented in his thesis by Simon van Endern was with a centralized approach with a distributed storage solution. The model can be seen in figure [X].
% Todo: Insert figure 
In his thesis, he eliminates the need for a central database, by collecting and storing raw data directly on the crowds's smartphone, creating a distributed database. This gives each participant a lot of control over their own data, just by deleting the application they can opt-out of future data analyses.

His original idea, the mobile phones would use peer-to-peer technology to forward the aggregation request and finally send the data to the server. But because of the lack of available technology, he opted to use the central server as an intermediary to send from mobile phone to mobile phone. To keep the data confidential and ensure anonymity from the server, he uses RSA and AES encryption. To forward the data to the next device, he implemented a polling solution. The application periodically polls the server for a new aggregation request targeted at the device. If one is present, it fetches the request over the REST API and after adding its own data posts it to the server.
In the finite scope, he managed to implement three aggregation types:

\begin{itemize}
	\item Average number of steps over the course of a day over all participants. 
	\item Average time spent on an activity (walking, running, in a vehicle or on a bicycle).
	\item Average number of steps of a participant during the test period. 
\end{itemize}

We examined his work and discovered a few flaws that we might be able to improve. We will look into possible solutions to make the architecture more efficient, scalable, secure and privacy-preserving. 

\subsection{Expanded Approach}
Simon van Endern's original idea, as is, can hardly be scaled because of the nature of its forwarding chain. Adding \(n\) new devices creates a linear time complexity of \(\mathcal{O}(n)\) and space complexity of \(\mathcal{O}(n^2)\). For every device added, the aggregation takes up to an additional 15 minutes and the data fetched and sent is the data of all previous devices combined, making it quite a burden on participant's data plan. 

To make the architecture more scalable, we propose to parallelize the chains, by building multiple groups for aggregation, as shown in figure [X]. With \(m\) groups, adding \(n\) devices, now only creates a time complexity of \(\mathcal{O}(\dfrac{n}{m})\) and space complexity of \(\mathcal{O}((\dfrac{n}{m})^2)\). So choosing the right \(m\) would change linear and quadratic to a potential constant growth. 

We also intend to implement differential privacy because ...
% Todo: more about differential privacy

Additionally to avoid unnecessary constant polling, we will explore some P2P frameworks that can send data directly to the next target device and cut down on forwarding delays.

In order to extend the aggregation options, we add the two following two types:
\begin{itemize}
	\item Number of participants that were at a location in a specific time period
	\item General location of all participants in a location at a specific time
\end{itemize}

