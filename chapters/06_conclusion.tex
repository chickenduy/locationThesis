% !TeX root = ../main.tex
% Add the above to each chapter to make compiling the PDF easier in some editors.

\chapter{Conclusion}\label{chapter:conclusion}
While it is hard to strike a balance between anonymity and usability, we think we found a starting point on which we can build for a more scalable, secure, and privacy-preserving platform for collecting, aggregating, and analyzing of mobility data.

\section{Research Questions}
\subsection*{RQ1: What are the benefits and drawbacks in Simon van Endern's architecture?}
After reexamination, we found that Simon van Endern managed to collect a lot of data while preserving privacy. Additionally, his idea provided a very secure communication infrastructure using a hybrid encryption scheme. His proposal, however, wasn't able to scale because of it, creating long wait times for data collection.
Furthermore, the single aggregation chain leads to to a quadratic growth in data consumption, and our current data plans can't carry that burden. On the upside, the lone series of collection hides the connection between the data and the device. On the downside, his implementation of forwarding requests using a polling system is also a drain on the battery.
Another flaw in the design is the selection process of the next device. Giving the server the ability to choose the following target allows it to choose compromised users and to endanger privacy. 

\subsection*{RQ2: What improvements are possible?}
In chapter 3, we proposed a few upgrades and extended further data collection. To help aggregation, we changed the polling mechanism to a straightforward event-driven one. Using a third-party push notification service, we are able to send requests directly to the participating mobile phones instead of giving them the job to fetch the requests from the server, improving significantly on communication time. We also suggested splitting the single aggregation chain into multiple groups, creating several smaller series of devices to collect the data instead of one long one, parallelizing the process, and cutting down on collection time. We also added the ability to request more accurate mobility data, such as the raw values of steps and activities and information about the location of devices or the presence of a device at a location. 

\subsection*{RQ3: Do the raw values for steps and activities infringe on privacy?}
We added raw values to the aggregation of steps and activities as we saw no infraction on privacy by providing these values if there are enough participants. It is crucial to expand the number of users to a certain number to provide privacy because as long as the numbers are low, we can trace the contribution of a participant in an aggregation. At a certain point, the steps and activities data should take the form of a uniform distribution, which then hides the individual raw data inside the collected data set. When the number of participants is reached and the inability to determine if a user participated in the collection request, we are confident that the privacy of the participants is secure. 

\subsection*{RQ4: Is it possible to query for more accurate location data while preserving privacy?}
We found that similar to other types of data, the number of participants is vital to protect their identity or at least their association. Using suppression, we are unable to hide devices, that are on their own. In chapter 5, we tested the platform in the field and a controlled setting. We were unable to draw any reliable conclusions regarding the usability of the collected data because of the lack of participants in our deployed field test. The privacy, however, we don't see any way how data can use to infer more sensitive data or identity. In the simulated test, we are able to collect much more reliable data. We were able to deduce the travel direction of a device but were unable to trace it back to other information.
% Todo: about location tracking and spacial cloaking

\subsection*{RQ5: Can we use the current state of P2P technology to remove intermediate third parties?}
Chapter 4 has shown our attempts to implement P2P technology into a mobile application. We have identified Textile as a viable candidate for our project, but unfortunately, one of the most important aspects didn't meet our expectations. There is a lot of progress in that technology, but some work still has to be done before it can be used in production. We should also look into decentralized communication standards, such as matrix, that can be used as an alternative to our initial direct communication approach.
% Todo: \cite[https://matrix.org/]

\section{Limitations}
As mentioned before, there are a lot of short-term obstacles that we identified that could be improved in the short-term and some that warrant further research in the long-term.

First, both our field test and simulated test were executed on a very limited number of devices, giving us only a glimpse in our solution to scalability and privacy. The number of participants in an aggregation request can range from a few people in a rural area to a few thousand or even a few million in crowded metropoles. Further tests should be performed to collect a more extensive data set, and we expect them to confirm our assumptions.

In chapter 3, for the sake of the scope of the thesis, we assumed that the server and the mobile devices are trusted, and a malicious actor only has access to the aggregated data. The possibility of a compromised server and compromised mobile phones must be taken into consideration. Given an untrusted server, it is able to view the groups and their participants and the data received from each group. We have suspicions that the server is able to connect the values sent to the mobile device it was sent from by resending the same request multiple times and observing the changes in the grouped collection. In case that there are compromised participants, the integrity of the data is in danger. The privacy of some devices might be jeopardized, but we believe that the mixed groups sent from the server cause enough change so that the malicious devices aren't able to build a profile. In the event that the server and participants are malicious, we aren't able to guarantee any protection.

\section{Future Work}
We have shown that the work we have done looks very promising, but a lot of issues were left untouched because of the limited constraints of the thesis. Below, we suggest a few subjects that could be considered in the future. 

\subsection{Mobile}
With the trend in decentralization, P2P technology is being developed hand-in-hand by many organizations. Being able to communicate with the next devices directly, instead of involving an intermediary, will cut out a possible point of failure and provide a more secure communication channel.

As discussed in the earlier chapters, homomorphic encryption could be an excellent tool to keep data confidential while enabling arithmetic operations on them. Using this method allows for the calculation of values without knowing the raw data, preventing sensitive information from prying eyes. This technology might not pertain to privacy and anonymity directly but would still make the platform more secure.

Another technique to strengthen privacy would be to give the users the control to participate in the data collection themselves. An option in which they can decide what level of privacy the collected data holds. I.e., either setting the collection to very private, so that all data gathered with the option is protected from aggregation or just changing the privacy level of the data itself.

Also mentioned in the section above is that trust is a topic that can't be handled lightly. Either intentionally providing false information or trying to trace or link private data to a device are possible motives for a malicious user. To combat misinformation and privacy leaks, there has to be a way to establish trust. M. Pouryazdan et al. suggest a score to measure the reputation, which in our eyes could either help with data reliability or create a quasi-identifier for an adversary to exploit.
% Todo: \cite[Quantifying User Reputation Scores, Data Trustworthiness, and User Incentives in Mobile Crowd-Sensing]

\subsection{Server}
Trust is also an issue for the server. With scandals constantly being reported on, the public trust in corporations and organizations collecting sensitive data is waning. Establishing the server as a trustworthy instance that doesn't mishandle the collected data by enhancing privacy or guaranteeing anonymity would be a step in the right direction. Relinquishing control over the data from a centralized architecture into a decentralized one could also help with this issue.

\subsection{Blockchain}
Leveraging the power of blockchain, the platform can incentivize the user to participate in aggregation. Ming Li et al. [X] suggest CrowdBC, a decentralized framework for crowdsourcing using blockchain. 

On the one hand, compensating the crowd for participating on the platform would result in more data collected. On the other hand, however, the existence of monetary value would draw in the attention of malicious intentions, making it hard for data to be trustworthy.

\section{Reproducibility}
Further research on the collection, aggregation, and analysis of mobility data with the use of a scalable, secure, and privacy-preserving platform can be done using our implementation. In the README file, we noted down instructions for the installation and local deployment of the server. The necessary tools for the setup are also documented. Comments in the code will indicate the places additional changes, such as API keys for the push notification service and URLs for the push notification server and the database instance, have to be added to run it correctly. As for storage, we built the platform with MongoDB's cloud solution in mind, so utilizing an alternative might require changes to the database module. Section [X] shows our deployment option, but the server can be hosted on any platform, and the database could be deployed locally.
% Todo: \cite[README]

On the client-side, we have mobile phones. The application can be built using the Android Studio IDE and installed on any compatible Android device. Similar to the server, the comments specify the locations where URLs have to be modified. The application was built from the ground up with modularity in mind, making it possible to replace communication schemes.
% Todo: \cite[application]

First, the server and the database has to be started. After that, the Android applications can be started, triggering the registration process automatically when the permissions have been accepted. With everything running, aggregation requests can be initiated by consuming the REST API detailed in section \ref{sec:hello}, and the end result should be stored in the deployed database instance.