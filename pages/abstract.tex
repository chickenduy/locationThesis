\chapter{\abstractname}
With a growing population crisis, urban planning can use mobility data to improve traffic, and infrastructure. But  most of the data is proprietary and controlled by the few companies that collect them and are unavailable to the public. It is not easy getting access and the publication of the data sets poses privacy risks that can have severe consequences. Furthermore, the privacy of the data itself relies on the intention of the companies that are collecting it. Our crowd sourcing platform uses Android phones to collect mobility data distributively and gives the public the ability to request data from the data collection. The approach publishes aggregated data in order to preserve the privacy of the user. We confirmed the viability of our proposed solution in a field test and a simulation. The results collected showed that aggregation of raw values from number of steps, times spent on an activity can be published without risking the privacy of the user. We also prove that spatially cloaked location data can be disclosed without violating privacy.